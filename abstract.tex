\chapter*{Abstract}\label{abstract}

\addcontentsline{toc}{chapter}{Abstract}

One of the biggest unsolved problems of the modern day is that of plasma confinement in fusion reactors; a challenge 
undertaken with the expressed intent of providing a safe, clean, reliable source of renewable energy. Various approaches 
to tackling this problem have emerged since the field's inception in 1946, but the most prominent design that has 
persisted through to modern day is the familiar donut-shaped Tokamak fusion reactor design. As equally as familiar as 
the quest for fusion power, are the challenges that face its achievement. The problem at the heart of plasma science is that 
of confinement, and one of the larger antagonists to this goal is the presence of runaway electron (RE) populations. 
These, if managed, can wreak havoc on the reactor, and disrupt the plasma - not conducive to confinement, and thus 
stability.

While most fusion reactors operate in what's known as ``DC'' mode, where the plasma current is oriented in a single direction, 
we focus our investigations instead on ``AC'' Tokamak operating modes, where the plasma current is allowed to oscillate 
back and forth. This, however, introduces its own set of problems - namely that of the observed presence of ``anti-parallel'' 
currents, characterised by excess populations of runaway electrons. 

Governing the behaviour of a plasma within the confines of a Tokamak's geometry is the Grad-Shafranov equation; a nonlinear, 
elliptic PDE derived from the principles of magnetohydrodynamics (MHD) - the union of Maxwell's equations and fluid dynamics. 
Solutions to this PDE provide magnetic flux information, and are instrumental to understanding the dynamics of a plasma, and thus, 
help inform answers to many of the challenging questions we face in fusion science. In this thesis we investigate a variation of 
the Grad-Shafranov equation specifically catering to current reversals, and seek to test whether we can simulate the ramp down phase 
of a current cycle, with the intent of identifying behaviour that might support the existence of anti-parallel currents, and/or 
provide explanation for the increased population of runaway electrons. We initially simulate this with emulated data, and then 
attempt to apply our model to data provided by the ISTTOK reactor.

In attempting to use our model on data from ISTTOK we found that, while there are some difficulties in 
fitting our parameters to the data that is available, initial results show promise that the model can be used to 
make observations about the problems described above. We conclude with some potential avenues for further research that 
could see the model improved.
