\chapter*{Abstract}\label{abstract}

\addcontentsline{toc}{chapter}{Abstract}

% Problem
%% Motivates and identifies crisply what problem you are facing
One of the biggest unsolved problems of the modern day is that of plasma confinement in fusion reactors; a challenge 
undertaken with the expressed intent of providing a safe, clean, reliable source of renewable energy \href{https://www.youtube.com/watch?v=dQw4w9WgXcQ}{[42]}. Various approaches 
to tackling this problem have emerged since the field's inception in 1946, but the most prominent design that has 
persisted through to modern day is the familiar donut-shaped Tokamak fusion reactor design. As equally as familiar as 
the quest for fusion power, are the challenges that prevent its realisation. The problem at the heart of plasma science is that 
of confinement, and one of the larger antagonists to this goal is the presence of runaway electron (RE) populations. 
These, if left unmanaged, can wreak havoc on the reactor, and disrupt the plasma - not conducive to stability, and thus 
not conducive to confinement. This problem is especially pertinent to the confinement of plasmas in AC design tokamaks, such as 
the ISTTOK project - the reactor which inspired this thesis. Studies are in disagreement on the presence of multiple, oppositely 
oriented current channels within a plasma, and the associated cause for the spikes in runaway electron populations during the 
quiescent phase of an AC cycle \cite{malaquias-matthew} - problems which directly affect the confinement of plasma in AC tokamaks. 

% Contribution / Results
In this thesis we look to model the behaviour of a system as the plasma current is inverted, with intent of providing 
a theoretical basis for verifying the work done by Hole and Malaquias \cite{malaquias-matthew}. We first show that we 
can approximate a time evolution of the Grad-Shafranov equation via small perturbations of equilibrium solutions, up to 
some $\epsilon$ term. From this, we use results afforded to us by Wang to find equilibrium solutions to a variant of the 
Grad-Shafranov equation which permits current reversal configurations \cite{wang-analytic-solution}. Upon build a model 
to represent these equilibrium, we simulate a current inversion and observe how the poloidal magnetic flux and pressure 
density profiles change. From here, we use our model to fit experimental data from ISTTOK, and comment on its effectiveness 
and what insight it provides us.

% Meaning
The work we've done here is a step towards understanding the dynamics of plasma in AC tokamak devices, with specific 
attention paid to the generation of runaway electrons. There are many ways in which our model can be extended and improved 
upon, which ultimately will lead to more accurate representations of the behaviour, providing more intriciate insight 
into these properties. More work can be done in correlating results with ISTTOK data, and in drawing more information 
from our simulations (e.g., electric field behaviour). 