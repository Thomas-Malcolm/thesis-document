
\chapter{Introduction}
\label{chapter1}

% 1.1.
\section{Plasma Science}

This thesis is multidisciplenary by nature, incorporating aspects of mathematics, physics and computer science throughout 
the various challenges faced in reaching the conclusions we have. 


Before burying ourselves in the thick of our results we 
will cover the requisite knowledge for working in the plasma science simulation space. First we'll discuss what 
the physical object of our attention (``plasma'') actually is, with a brief discussion on the design of fusion reactors, where we will 
emphasise the structures that specifically relate to our interests. In the background chapter we will expound on this by 
building the mathematical theory underpinning physical processes within the fusion reactor, providing us a way to reason
about the behaviour of plasma (and related processes) inside a fusion reactor, or, more accurately, approximate their behaviour 
via simulation.




% 1.1.1.
\subsection{I'm a Mathematician... what is ``plasma''?}

While an initially daunting topic, fear not fearful mathematician, for many of the inherently physical 
behaviours in plasma science we require can be expressed in terms of our dearest mathematical expressions -- differential equations! 
But first, what actually is ``plasma''? Webster's dictionary defines plasma to be ``a green faintly translucent quartz'' \cite{websters_plasma}. 
While I'm sure there isn't no relation between crystals and our investigations, this is unfortunately, not the recipient of our affection. 

When plasma is referred to in everyday conversation it is often noted as being the ``4th state of matter''. To introduce slightly more rigour,
plasma is an extension of the gaseous state of matter, where its energy (read: temperature) is increased sufficiently high that the electrons 
are no longer bound by the electromagnetic force to the atom's nucleus REFERENCE. The resulting substance is a ``pool'' of cations (the positively charged nuclei), and 
electrons (negatively charged), that exhibits interesting properties. It is these properties that we seek to exploit in the search of controlled, 
sustained fusion reactions.

Plasma is abundant in nature -- just not in many places that we as Humans commonly look. Stars are the most immediate example of matter in a plasma state, 
and are readily viewable (at least for half the day). Lightning strikes are paths through the atmosphere which are ionised, 
and neon signs work by heating Neon gas within a tube to ionise it REFERENCE.

The question then is, what is ``fusion'', and how does this relate to plasma? The answer comes back to energy. Analogously to a fission reaction, 
where energy is released through the division of atoms, one can also fuse two separate atoms together and have large amounts of energy emitted 
as a bi-product of that fusion. It is (essentially) this extra energy that we wish to harness when harvesting energy from a fusion reactor (which we'll 
discuss in the next part). When two 

HYDROGEN FUSION EQUATION

How do two hydrogen atoms fuse? This process can occur most easily when in a plasma state, as repellant forces are minimised.


% 1.1.2.
\subsection{Fusion Reactor Design 101}

Diagram

Poloidal vs toroidal flux

Magnet positioning

Heating of plasma

Confinement

REs

% 1.2
\section{What problem does this thesis address?}

Talk about paper Matthew and Artur released. Presence of residual REs.

% 1.2.1
\subsection{What results do we seek?}

To provide a theoretical basis for exploring the observed anomalies. 
To see if theory supports the observation

% 1.3.
\section{Project Progression}

Work inspired by Matthew's paper 

First expanded MHD equations linearly with a perturbation treatment

Mathematical PDE theory developed by Wang. Identified errors in paper, fixed. 

Developed a simulation to reproduce results. Then went other direction, 
solving for parameters for data. 

Simulated current inversion.

ISTTOK data matching.

Feasibility



% 1.4.
\section{Structure of Thesis}

TODO (after writing it)

