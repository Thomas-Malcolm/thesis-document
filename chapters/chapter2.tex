
\chapter{Background}
\label{chapter2}

% 2.1
\section{Magnetohydrodynamics}


The term ``magnetohydrodynamics'' (MHD) is a portmanteau of two physical concepts which are used to model plasmas 
inside fusion reactors: the ``magneto'' term comes from ``magnetic field'', and ``hydrodynamics'' indicates a 
a fluid dynamics component. Put together, magnetohydrodynamics is the study of electrically conductive materials 
that behave like fluids. Essentially it provides a way to model the behaviour of a (considerably volumous!) 
mass of particles, and their electrodynamic forces, as if it were a fluid, as opposed to having to model 
individual particle interactions. This idea was first introducd by Hannes Alfv\'en in 1970, for which he earned 
the Nobel prize \cite{alfven-mhd}!

Much modern research uses some variation of an MHD model (and the derived Grad-Shafranov Equation, which we will 
soon be introduced to) for a number of reasons, not least being its comparative computational efficiency.
One primary benefit of treating our plasma as a fluid being that we avoid modelling the behaviour of 
each individual particle in said plasma, a simplification which becomes especially important when we consider 
the order of number of particles we would have to simulate is of order $\sim 10^{20}$ - far too much 
for the author's ThinkPad to even contemplate!

Here we will build the relevant MHD background for this thesis, deriving the MHD equations from first principles, 
and explaning the assumptions we make to reduce them to a simplified state known as ``ideal MHD''. From there we will 
look at the PDE which models the behaviour of a plasma inside a Tokamak, the ``Grad-Shafranov Equation''. We will 
then note some pitfalls of using this model to describe AC configuration Tokamaks (as we are investigating), 
and finish with a discussion on runaway electrons (RE).

% 2.1.1.
\subsection{MHD Theory}

Given MHD is the marriage of fluid dynamics with electrodynamics, it is only natural to begin our 
study looking at the equations which describe electrodynamic behaviour --- Maxwell's equations 
describe the interaction between magnetic fields $\vec{B}$, electric fields $\vec{E}$, and 
the current density $j$ which induces them. Thus, we introduce Maxwell's equations:
\begin{definition}
    Maxwell's equations are given \cite{wesson-tokamaks}:
    \begin{align}
        \nabla \times \vec{B} &= \mu_0 j + \frac{1}{c^2} \pdv{\vec{E}}{t} \\
        \nabla \times E &= -\pdv{\vec{B}}{t} \\
        \nabla \cdot \vec{B} &= 0 \\
        \nabla \cdot E &= \frac{\rho_c}{\epsilon_0}
    \end{align}

    where $\mu_0$ is free-space magnetic permeability (in $\text{henry}\; m^{-1}$), and $c$ is the speed of light.
    
\end{definition}

Lagrangian operator

Maxwell's equations

Assumptions and simplifications (leading into ideal MHD)

% 2.1.2.
\subsection{Ideal MHD}

Simplifications made to derive ideal MHD equations 

Use in GS equation 

Physical implications

% 2.2. 
\section{Grad-Shafranov Equation}

What is it

What do its components describe

Solutions to GS equation 

% 2.2.1
\subsection{Derivation}

Go through derivation of it 

% 2.3.
\section{AC Configuration Tokamaks}

Tokamak's normally in DC mode. What does it mean for a fusion reactor 
to operate in AC mode

% 2.3.1.
\subsection{Physical Differences}

Plasma current (talk: operation of toroidal coils to regulate current)



% 2.3.2.
\subsection{Why Bother?}

What benefits are there to using an AC design over DC?

Talk about confinement time, stability, etc

% 2.4
\subsection{Runaway Electrons}

\subsubsection{Generation Mechanisms}

Here we will discuss how runaway electrons come to exist within a Tokamak's fusion cycle.

\subsubsection{RE Detection}

\subsubsection{Why we care}

