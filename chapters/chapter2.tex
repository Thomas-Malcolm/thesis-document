
\chapter{Background}
\label{chapter2}

% 2.1
\section{Magnetohydrodynamics}


The term ``magnetohydrodynamics'' (MHD) is a portmanteau of two physical concepts which are used to model plasmas 
inside fusion reactors: the ``magneto'' term comes from ``magnetic field'', and ``hydrodynamics'' indicates a 
a fluid dynamics component. Put together, magnetohydrodynamics is the study of electrically conductive materials 
that behave like fluids. Essentially it provides a way to model the behaviour of a (considerably volumous!) 
mass of particles, and their electrodynamic forces, as if it were a fluid, as opposed to having to model 
individual particle interactions. This idea was first introducd by Hannes Alfv\'en in 1970, for which he earned 
the Nobel prize \cite{alfven-mhd}!

Much modern research uses some variation of an MHD model (and the derived Grad-Shafranov Equation, which we will 
soon be introduced to) for a number of reasons, not least being its comparative computational efficiency.
One primary benefit of treating our plasma as a fluid being that we avoid modelling the behaviour of 
each individual particle in said plasma, a simplification which becomes especially important when we consider 
the order of number of particles we would have to simulate is of order $\sim 10^{20}$ - far too much 
for the author's ThinkPad to even contemplate!

Here we will build the relevant MHD background for this thesis, deriving the MHD equations from first principles, 
and explaning the assumptions we make to reduce them to a simplified state known as ``ideal MHD''. From there we will 
look at the PDE which models the behaviour of a plasma inside a Tokamak, the ``Grad-Shafranov Equation''. We will 
then note some pitfalls of using this model to describe AC configuration Tokamaks (as we are investigating), 
and finish with a discussion on runaway electrons (RE).

% 2.1.1.
\subsection{MHD Theory}

Given MHD is the marriage of fluid dynamics with electrodynamics, it is only natural to begin our 
study looking at the equations which describe electrodynamic behaviour --- Maxwell's equations 
describe the interaction between magnetic fields $\vec{B}(\vec{r}, t)$, electric fields $\vec{E}(\vec{r}, t)$, and 
the current density $j(\vec{r}, t)$ which induces them, where $\vec{r} \in \R^3$ is a position vector and $t \in \R$ describes time. Thus, we introduce Maxwell's equations:
\begin{definition}
    Maxwell's equations are given \cite{wesson-tokamaks}:
    \begin{align}
        \nabla \times \vec{B} &= \mu_0 j + \frac{1}{c^2} \pdv{\vec{E}}{t} \\
        \nabla \times E &= -\pdv{\vec{B}}{t} \\
        \nabla \cdot \vec{B} &= 0 \\
        \nabla \cdot E &= \frac{\rho_c}{\epsilon_0}
    \end{align}

    The functions driving change in this system are $\rho_c(\vec{r}, t)$, the electric charge density, and $j(\vec{r}, t)$, 
    the electric current density. We also have $\mu_0$, the free-space magnetic permeability (in henry $m^{-1}$); $\epsilon_0$, 
    the free-space permittivity; and $c$, the speed of light.
    
\end{definition}

These equations give us a way to reason about the electric and magnetic fields if we're given some descriptor for the 
current we're passing through some medium. We are now to introduce the fluid dynamics component to our system. A fluid's 
mass density can be given by summing over the effects of individual ``species'' of particles (e.g. electrons) in the fluid:
$$\rho_c = \sum_{\sigma} m_{\sigma} n_{\sigma}$$
and its current density similarly:
$$\vec{j}(\vec{r}, t) = \sum_{\sigma} n_{\sigma} q_{\sigma} \vec{u}_{\sigma}$$
where $\sigma$ describes a particle species, $m_\sigma$ describes its mass, $n_\sigma$ its number density 
(a measurement of concentration for the given particle species in a pre-defined volume -- akin to Avogadro's constant), 
$q_\sigma$ describes its electric charge, and $\vec{u}_\sigma$ the mean velocity of this species of particle in the fluid. 

\subsubsection{Fluid Dynamics}

\begin{notn}
    A common simplification in notation for fluids is made in using the \emph{Lagrangian derivative}, given:
    $$\frac{\DD}{\DD t} = \left (\pdv{t} + \vec{u} \cdot \nabla \right )$$ 
    It describes the total change in a volume within a fluid as it moves throughout said fluid. It is essentially 
    a change in reference frame for a derivative - where a regular derivative might descirbe, for example, 
    how a particle moves with respect to time in its surroundings, its Lagrangian will take into account 
    the motion of the fluid the particle is immersed in as well.
\end{notn}

We begin with conservation of mass, also known as the ``continuity equation''. 

\begin{definition}[The continuity equation]
    The below relates how mass density, $\rho$, changes with respect to the motion of a 
    fluid element.

    \begin{equation}
        \pdv{\rho}{t}  = -\nabla \cdot \left ( \rho \vec{u} \right ) \label{continuity}
    \end{equation}

    The derivation of the above comes from a surface integral over a volume with an outward and inward flux, 
    and an application of Gauss' flux law. For a full derivation, see pg. 19 - 21 of \cite{mhd-lectures}.
\end{definition}

\begin{remark}
    The above is a PDE with four variables: $\rho$, the mass density of the medium, and $\vec{u}$, the velocity 
    of the fluid. This renders the system not closed, and thus too general for an analytic solution - we have 
    more unknowns than we have equations \cite{mhd-lectures}. Later we 
    will introduce other equations to our system to apply more restrictions, and make assumptions about the 
    physicality of the system which will reduce these dependencies, and make it determined (``closed'').
\end{remark}

\begin{remark}
    We can rewrite \eqref{continuity} with a Lagrangian frame of reference, as 
    \begin{align}
        &\frac{\DD}{\DD t}\rho = -\rho \nabla \cdot \vec{u} \\
        \iff &\frac{\DD}{\DD t}\rho + \rho \nabla \cdot \vec{u} = 0 \label{continuity-lag}
    \end{align}
\end{remark}

Equation \eqref{continuity} (and equivalently \eqref{continuity-lag}) tells us that the mass of our fluid is conserved for motion of a volume element 
of our fluid - one assumption we make for our model. Next we'll discuss fluid motion as described by Newton for a fluid element:

\begin{definition}[Newtonian Fluid Motion]
    Newton's law for a fluid specifies:
    \begin{align}
        \rho \frac{\DD }{\DD t} \vec{u} &= \vec{F}
    \end{align}
    where $\rho(\vec{r}, t)$ is the mass density of the fluid, 
    $\vec{u}(\vec{r}, t)$ describes the velocity of the fluid element, and $\vec{F}(\vec{r}, t)$ 
    describes the force per unit volume acting on the fluid element \cite{mhd-lectures}.
\end{definition}
The forces acting on particles within a fluid can be split into two types
\begin{itemize}
    \item Gravitational
    
    Here, $\vec{F}_g = \rho \vec{g}$, where $\vec{g}$ is the gravitational acceleration. This should hark back to high school 
    physics, though note this is a vector here as we care about the direction gravity accelerates the fluid element in, 
    and relativistic effects can be an important consideration for high mass systems. This is more relevant for cases that 
    you are using the MHD equations to describe the dynamics of large systems, such as a star. Unsurprisingly, this is less relevant 
    for our case of plasmas within relatively miniscule Tokamaks.

    \item Electromagnetic
    
    This is the interesting part for us. As we assume our fluid is capable of conducting electricity (it's a plasma after all),
    there are electromagnetic forces operating within the fluid that affect the behaviour of the particles that the fluid consists of. 
    The electromagnetic forces themselves can be split into two types, the \textbf{electric force} given by $\vec{F}_q = \rho_c \vec{E}$, 
    and the \textbf{Lorentz force}, given $\vec{F}_{L} = \vec{j} \times \vec{B}$ (where $\vec{B}(\vec{r}, t)$ describes the magnetic field)
\end{itemize}
Taking these forces into account, we can describe the motion of an element of our fluid moving with velocity $\vec{u}$ 
via:
\begin{equation}
    \rho \frac{\DD}{\DD t} \vec{u} =  \vec{j} \times \vec{B} + \rho_c \vec{E} - \nabla p + \rho \vec{g}
\end{equation}


\begin{remark}
    Here the $\rho \vec{g}$ term could be abstracted further away into a stress tensor, as described by equation 4.20 of \cite{mhd-lectures}. These 
    pressures however are largely negligible when dealing with the scale we do in Tokamak plasmas however, and 
    are thus ignored. We will soon drop the gravitational consideration as well anyway, but include it here for now for completeness.
\end{remark}

\begin{remark}
    The equation we have introduced is a function of six variables in its complex form (with stress tensor 
    included), though even in this form we still have more variables than we do equations (1). Similar to before, this 
    is not constrained sufficiently to consider it a closed system.
\end{remark}

We next relate a plasma's pressure to its motion. We will simply present it here, though important notes in its 
derivation are that we assume the plasma behaves as an ideal gas (which is to say the only interaction 
between particles within the plasma are via elastic collisions with each other, or the boundaries of the container 
it is contained within). This is equivalent to saying that energy in the system depends only on the pressure. Thus, the 
energy equation is given:
\begin{definition}[Energy Equation]
    Where $\vec{p}(\vec{r}, t)$ describes the pressure of our fluid:
    \begin{equation}
        \frac{\DD}{\DD t} p  = -\Gamma p \nabla \cdot \vec{u} + (\Gamma - 1) \left [ -\nabla \cdot \vec{q} + \vec{\Pi} : \nabla \vec{u} + \eta \vec{J}^2 \right ]
    \end{equation}
    where $\Gamma$ describes ``abiabatic index'' (a known constant for plasmas), $q$ is the heat flux through 
    the boundary of the volume; $\eta$ is the electrical resistivity of the fluid; and $\vec{\Pi}$ is the viscous 
    stress tensor (the component which we replaced with $\rho \vec{g}$ earlier), and will soon ignore again. 
\end{definition}



The equations we've looked at constitute what are known as the fluid equations:

\begin{definition}[Fluid Equations]
    \begin{align}
        \frac{\DD}{\DD t} \rho &+ \nabla  \cdot \rho \vec{u} = 0 \\
        \rho \frac{\DD}{\DD t} \vec{u} &=  \vec{j} \times \vec{B} + \rho_c \vec{E} - \nabla p + \rho \vec{g} \\
        \frac{\DD}{\DD t} p  &= -\Gamma p \nabla \cdot \vec{u} + (\Gamma - 1) \left [ -\nabla \cdot \vec{q} + \vec{\Pi} : \nabla \vec{u} + \eta \vec{j}^2 \right ]
    \end{align}
\end{definition}

As reiterated a couple times now, these equations form an unclosed system, and are thus undetermined. 
To resolve this we introduce some constraints that come from electrodynamic forces, and a couple other 
relations that lead to a closed system.

\subsubsection{Electrodynamics}

As it currently stands, the input variables for the fluid equations are $\rho(\vec{r}, t)$ and $p(\vec{r}, t)$. We note that 
the electric field $\vec{E}$ and the magnetic field $\vec{B}$ are generated by the electric charge density, $\rho_c$, and 
the current density $\vec{j}$. This is where Maxwell's equations come into play. By combining Maxwell's equations with the 
fluid equation given above, we achieve the MHD model. The only piece to our puzzle missing is to tie the motion of the fluid 
(through $\vec{u}$) to the behaviour of the electric and magnetic fields. This is done via \textit{Ohm's} law:
\begin{equation}
    \vec{E} + \vec{u} \times \vec{B} = \eta \vec{j}
\end{equation}
\begin{remark}
    Note that the above is technically a lie, as it does not take into account relativistic effects, though for simplicity 
    our MHD model ignores these.
\end{remark}


\begin{definition}[MHD Equations]
    The MHD equations can then be summarised:
    \begin{align}
        \frac{\DD}{\DD t} \rho &= 0 \\
        \rho \frac{\DD}{\DD t} \vec{u} &= -\nabla p + \vec{j} \times \vec{B} + \nabla \cdot \Pi \\
        \frac{\DD}{\DD t} p  &= -\Gamma p \nabla \cdot \vec{u} + (\Gamma - 1) \left [ -\nabla \cdot \vec{q} + \vec{\Pi} : \nabla \vec{u} + \eta \vec{j}^2 \right ] \\
        \pdv{\vec{B}}{t} &= -\nabla \times \vec{E} \\
        \mu_0 \vec{j} &= \nabla \times \vec{B} \\
        \vec{E} + \vec{u} \times \vec{B} &= \eta \vec{j}
    \end{align}
\end{definition}

\begin{remark}
    The MHD equations as presented above constitute 14 equations with 27 unknowns. The breakdown is as such:
    \begin{itemize}
        \item $\rho$ : 1 unknown
        \item $\vec{u}$ : 3 unknowns
        \item $p$: 1 unknown
        \item $\vec{\Pi}$ : 9 unknowns
        \item $\vec{j}$ : 3 unknowns
        \item $\vec{B}$ : 3 unknowns
        \item $\vec{E}$ : 3 unknowns
        \item $\vec{q}$ : 3 unknowns
        \item $\eta$ : 1 unknown
    \end{itemize}
\end{remark}

The above is obviously insufficiently constrained for purposes of identifying a solution. We will skip a large amount 
of the work required to reduce the above to a closed system, though for details see lecture 7 of \cite{mhd-lectures}. 
For now, we will comment on two reduced MHD models:

\subsubsection{Resistive MHD}
The resistive MHD model comes about by setting $\vec{q} = 0$ and $\vec{\Pi} = 0$. Here, we have $\eta \ne 0$ notably. The model 
is given:
\begin{definition}[Resistive MHD]
    \begin{align}
        \frac{\DD}{\DD t} \rho &= 0 \\
        \rho \frac{\DD}{\DD t} \vec{u} &= -\nabla p + \vec{j} \times \vec{B}\\
        \frac{\DD}{\DD t} p  &= -\Gamma p \nabla \cdot \vec{u} \\
        \pdv{\vec{B}}{t} &= -\nabla \times \vec{E} \\
        \mu_0 \vec{j} &= \nabla \times \vec{B} \\
        \vec{E} + \vec{u} \times \vec{B} &= \eta \vec{j}
    \end{align}
\end{definition}

\begin{remark}
    The most notable effect of resistive MHD is that allowing for electrons to diffuse allows the resulting magnetic field lines 
    to reconnect, which leads to breaks in the magnetic field line topology. This can lead to the generation of fast particles, i.e., 
    runaway electrons.
\end{remark}


\subsubsection{Ideal MHD}
The ideal MHD equations are one step removed from the resistive MHD model --- in fact they are equivalent, only for the ideal MHD case 
we also ignore resistivity. Thus, set $\eta = 0$, and we obtain the ideal MHD equations:
\begin{definition}[Ideal MHD]
    \begin{align}
        \frac{\DD}{\DD t} \rho &= 0 \\
        \rho \frac{\DD}{\DD t} \vec{u} &= -\nabla p + \vec{j} \times \vec{B}\\
        \frac{\DD}{\DD t} p  &= -\Gamma p \nabla \cdot \vec{u} \\
        \pdv{\vec{B}}{t} &= -\nabla \times \vec{E} \\
        \mu_0 \vec{j} &= \nabla \times \vec{B} \\
        \vec{E} + \vec{u} \times \vec{B} &= 0
    \end{align}
\end{definition}

\begin{remark}
    In doing this, we have removed the dependency on $\vec{\Pi}$, $\eta$ and $\vec{q}$, which accounts 
    for 13 unknowns. This brings the total number of equations to 14 (or $8$ if you make a substitution 
    for Ohm's law) with 14 ($8$) unknowns. Thus, under the ideal MHD model, the system is closed.
\end{remark}


\red{Brief discussion on what resistivity is?}
\red{Tie into GSE}
% 2.2. 
\section{Grad-Shafranov Equation}

What is it

What do its components describe

Solutions to GS equation?

% 2.2.1
\subsection{Derivation}

Go through derivation of it 

% 2.3.
\section{AC Configuration Tokamaks}

Tokamak's normally in DC mode. What does it mean for a fusion reactor 
to operate in AC mode

% 2.3.1.
\subsection{Physical Differences}

Plasma current (talk: operation of toroidal coils to regulate current)



% 2.3.2.
\subsection{Why Bother?}

What benefits are there to using an AC design over DC?

Talk about confinement time, stability, etc

% 2.4
\subsection{Runaway Electrons}

\subsubsection{Generation Mechanisms}

Here we will discuss how runaway electrons come to exist within a Tokamak's fusion cycle.

\subsubsection{RE Detection}

\subsubsection{Why we care}

