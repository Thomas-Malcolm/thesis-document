
\chapter{Background}
\label{chapter2}

% 2.1
\section{Magnetohydrodynamics}


The term ``magnetohydrodynamics'' (MHD) is a portmanteau of two physical concepts which are used to model plasmas 
inside fusion reactors: the ``magneto'' term comes from ``magnetic field'', and ``hydrodynamics'' indicates a 
a fluid dynamics component. Put together, magnetohydrodynamics is the study of electrically conductive materials 
that behave like fluids. Essentially it provides a way to model the behaviour of a (considerably volumous!) 
mass of particles, and their electrodynamic forces, as if it were a fluid, as opposed to having to model 
individual particle interactions. This idea was first introducd by Hannes Alfv\'en in 1970, for which he earned 
the Nobel prize \cite{alfven-mhd}!

Much modern research uses some variation of an MHD model (and the derived Grad-Shafranov Equation, which we will 
soon be introduced to) for a number of reasons, not least being its comparative computational efficiency.
One primary benefit of treating our plasma as a fluid being that we avoid modelling the behaviour of 
each individual particle in said plasma, a simplification which becomes especially important when we consider 
the order of number of particles we would have to simulate is of order $\sim 10^{20}$ - far too much 
for the author's ThinkPad to even contemplate!

Here we will build the relevant MHD background for this thesis, deriving the MHD equations from first principles, 
and explaning the assumptions we make to reduce them to a simplified state known as ``ideal MHD''. From there we will 
look at the PDE which models the behaviour of a plasma inside a Tokamak, the ``Grad-Shafranov Equation''. We will 
then note some pitfalls of using this model to describe AC configuration Tokamaks (as we are investigating), 
and finish with a discussion on runaway electrons (RE).

% 2.1.1.
\subsection{MHD Theory}

Given MHD is the marriage of fluid dynamics with electrodynamics, it is only natural to begin our 
study looking at the equations which describe electrodynamic behaviour --- Maxwell's equations 
describe the interaction between magnetic fields $\vec{B}(\vec{r}, t)$, electric fields $\vec{E}(\vec{r}, t)$, and 
the current density $j(\vec{r}, t)$ which induces them, where $\vec{r} \in \R^3$ is a position vector and $t \in \R$ describes time. Thus, we introduce Maxwell's equations:
\begin{definition}
    Maxwell's equations are given \cite{wesson-tokamaks}:
    \begin{align}
        \nabla \times \vec{B} &= \mu_0 j + \frac{1}{c^2} \pdv{\vec{E}}{t} \\
        \nabla \times E &= -\pdv{\vec{B}}{t} \\
        \nabla \cdot \vec{B} &= 0 \\
        \nabla \cdot E &= \frac{\rho_c}{\epsilon_0}
    \end{align}

    The functions driving change in this system are $\rho_c(\vec{r}, t)$, the electric charge density, and $j(\vec{r}, t)$, 
    the electric current density. We also have $\mu_0$, the free-space magnetic permeability (in henry $m^{-1}$); $\epsilon_0$, 
    the free-space permittivity; and $c$, the speed of light.
    
\end{definition}

These equations give us a way to reason about the electric and magnetic fields if we're given some descriptor for the 
current we're passing through some medium. We are now to introduce the fluid dynamics component to our system. A fluid's 
mass density can be given by summing over the effects of individual ``species'' of particles (e.g. electrons) in the fluid:
$$\rho_c = \sum_{\sigma} m_{\sigma} n_{\sigma}$$
and its current density similarly:
$$\vec{j}(\vec{r}, t) = \sum_{\sigma} n_{\sigma} q_{\sigma} \vec{u}_{\sigma}$$
where $\sigma$ describes a particle species, $m_\sigma$ describes its mass, $n_\sigma$ its number density 
(a measurement of concentration for the given particle species in a pre-defined volume -- akin to Avogadro's constant), 
$q_\sigma$ describes its electric charge, and $\vec{u}_\sigma$ the mean velocity of this species of particle in the fluid. 


\begin{notn}
    A common simplification in notation for fluids is made in using the \emph{Lagrangian derivative}, given:
    $$\frac{\DD}{\DD t} = \left (\pdv{t} + \vec{u} \cdot \nabla \right )$$ 
    It describes the total change in a volume within a fluid as it moves throughout said fluid. It is essentially 
    a change in reference frame for a derivative - where a regular derivative might descirbe, for example, 
    how a particle moves with respect to time in its surroundings, its Lagrangian will take into account 
    the motion of the fluid the particle is immersed in as well.
\end{notn}

We begin with fluid motion as described by Newton:
\begin{definition}
    Newton's law for a fluid:
    \begin{align}
        \rho \frac{\DD }{\DD t} \vec{V} &= \vec{F}
    \end{align}
    where $\rho(\vec{r}, t)$ is the mass density of the fluid, 
    $\vec{V}(\vec{r}, t)$ describes the velocity of the fluid element, and $\vec{F}(\vec{r}, t)$ 
    describes the force per unit volume acting on the fluid element \cite{mhd-lectures}.
\end{definition}
The forces acting on a fluid can be split into two cases
\begin{itemize}
    \item Gravitational
    
    Here, $\vec{F}_g = \rho \vec{g}$, where $\vec{g}$ is the gravitational acceleration. This should hark back to high school 
    physics, though note this is a vector here as we care about the direction gravity accelerates the fluid element in, 
    and relativistic effects can be an important consideration for high mass systems. This is more relevant for cases that 
    you are using the MHD equations to describe the dynamics of large systems, such as a star. Unsurprisingly, this is less relevant 
    for our case of plasmas within relatively miniscule Tokamaks.

    \item Electromagnetic
    
    This is the interesting part for us. As we assume our fluid is capable of conducting electricity (it's a plasma after all),
    there are electromagnetic forces operating within the fluid that affect the behaviour of the particles that the fluid consists of. 
    The electromagnetic forces themselves can be split into two types, the \textbf{electric force} given by $\vec{F}_q = \rho_c \vec{E}$, 
    and the \textbf{Lorentz force}, given $\vec{F}_{L} = \vec{j} \times \vec{B}$ (where $\vec{B}(\vec{r}, t)$ describes the magnetic field)
\end{itemize}
Taking these forces into account, we can describe the motion of an element of our fluid moving with velocity $\vec{u}$ 
via:
\begin{equation}
    \rho \frac{\DD}{\DD t} \vec{u} = \rho_c \left ( \vec{j} \times \vec{B} + \vec{E} \right ) - \nabla p + \rho \vec{g}
\end{equation}


\begin{definition}
    Fluid equations:
    \begin{align}
        \frac{\DD}{\DD t} \rho + \rho \nabla \cdot \vec{u} &= \pdv{\rho}{t} + \nabla \cdot (\rho \vec{u}) = 0 \\
        \frac{\DD}{\DD t} p + \gamma p \nabla \cdot \vec{u} &= \pdv{p}{t} + \vec{u} \cdot \nabla p + \gamma p \nabla \cdot \vec{u} = 0
    \end{align}
\end{definition}



These equations are quite volumous, and are too 
cumbersome for our purposes in terms of computation, and in the complexity they create as a system of PDEs.
We then wish to make a number of assumptions to simplify them to a point at which they are usable. The next 
section will discuss what assumptions and simplifications we make, and the concessions which come with that.

% 2.1.2.
\subsection{Ideal MHD}

Simplifications made to derive ideal MHD equations 

Use in GS equation 

Physical implications

% 2.2. 
\section{Grad-Shafranov Equation}

What is it

What do its components describe

Solutions to GS equation 

% 2.2.1
\subsection{Derivation}

Go through derivation of it 

% 2.3.
\section{AC Configuration Tokamaks}

Tokamak's normally in DC mode. What does it mean for a fusion reactor 
to operate in AC mode

% 2.3.1.
\subsection{Physical Differences}

Plasma current (talk: operation of toroidal coils to regulate current)



% 2.3.2.
\subsection{Why Bother?}

What benefits are there to using an AC design over DC?

Talk about confinement time, stability, etc

% 2.4
\subsection{Runaway Electrons}

\subsubsection{Generation Mechanisms}

Here we will discuss how runaway electrons come to exist within a Tokamak's fusion cycle.

\subsubsection{RE Detection}

\subsubsection{Why we care}

