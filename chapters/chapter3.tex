
\chapter{Current Reversal Theory}
\label{chapter3}



% 3.1.
\section{Perturbation Methods}

% 3.1.1.
\subsection{What is a Perturbation Treatment?}

When it comes to solving differential equations, the ideal scenario 
is that you find an analytic solution, $\Psi$. This, however, is generally a 
difficult problem, and especially so in the case of differential equations. A great deal 
of research goes into trying to find analytic solutions to PDEs, 
which is no less true for the Grad-Shafranov equation and its variations. Though 
such efforts are often futile, or require making assumptions about properties of 
your solution which may not be representative of what you're trying to show, 
other means of obtaining the $\Psi(\vec{x})$ for $\vec{x} \in \Omega$ have been developed.

A traditional approach to this is to use some numerical code to approximate solution values. 
There are various codes which exist for this purpose specifically with regards to 
the GSE equation, such as HELENA REFERENCE. These codes come in many flavours - 
some implement a particle simulation, which can provide precision with simulation, 
though are computationally costly. Others exploit various properties of the 
variant of GSE they target and may be computationally more feasible, but are 
restricted in scope, application, and/or reliability because of their assumptions.

A common problem that numerical methods seek to solve is that of time evolution. 
Simulations which evolve a system through time, for obvious reasons, do not discard 
the time dependency component in their derivation of the GSE. This, however, often 
makes them computationally expensive as a result, while also increasing the difficulty of 
finding analytic solutions to the system. 

For this thesis, given the availability of an analytic solution to a variation of 
the GSE that specifically relates to current reversals, we employ a hybrid approach. 
In our simulation we wish to observe changes in the system through a current ramp down
- this necessitates a time dependency component to our solution. However, there are 
a couple points to note:
\begin{itemize}
    \item Time evolution simulations are expensive
    \item Existing literature on time evolution GSE do not take into account the possibility for current 
    reversal (e.g. resistivity)
    \item The GSH variant we utilise (which has yet to be introduced) is not time dependent, i.e., 
    was not derived taking into account a time component
\end{itemize}

There are a couple possible avenues we could explore from here. We could implement a numerical code 
for the time evolution, for example a particle simulation, however this is for all intents and purposes
computationally infeasible given the hardware available (a beloved but dilapidated Thinkpad T440p). We could 
attempt to find an analytic solution to the GSE with time dependence (or an approximation to one),
but given the mountain of research existing in this space and the relative inexperience 
of the author, this would likely be a futile task. The third, and most lucrative option, 
is to use an existing analytic solution to a variation of the GSE that accounts for 
current reversal, and explore if small changes to its state
are sufficiently ``good enough'' approximations of .

This is where perturbation theory comes in. Intuitively we would expect a 
plasma to vary ``smoothly'' - nature rarely behaves in instantaneous ways. The 
analytic solution we have describes the instantaneous state (a slice in time) of 
our reactor's state. We may anticipate then that if we were to change something 
in our system by a ``sufficiently small amount'', that our system may react 
to that change of state in a way that approximates how it would if it 
were really varying through time. We will have the ability to determine the 
state of our system for some given parameters, and so the question becomes: can 
we vary our input in a way such that the behaviour of the analytic solution to 
the GSH is an approximation of a time evolution of the plasma. We are able to compare 
our results to experimental data, and then exploit our work to answer further questions.



% 3.1.2.
\subsection{Regular Perturbation Theory}


\url{https://jmahaffy.sdsu.edu/courses/f19/math537/beamer/perturb.pdf}
\url{https://www.ucl.ac.uk/~ucahhwi/LTCC/section2-3-perturb-regular.pdf}
\url{https://math.byu.edu/~bakker/Math521/Lectures/M521Lec15.pdf}

% 3.1.3.
\subsection{Ideal MHD Perturbation}

At the beginning of this thesis we introduced it as being interdisciplenary in nature. 
Keeping in that spirit, at times here we will put on our physicists cap and 
seemingly arbitrarily remove terms we no longer wish to have. We 
kindly ask that any mathematicians reading these sections avert their attention 
so as to maintain sanity.

In Chapter \ref{chapter2} we derived the ideal MHD equations, given:
\begin{align}
    j \times B &= \nabla p \\
    \mu_0 j &= \nabla \times B \\
    \nabla \dot B &= 0
\end{align}
These, notably, are the time independent variation. We wish to look at the 
effect of a small ($\epsilon > 0$) time perturbation to this system, and so 
will return to this derivation, without discarding the time component. We will 
begin with Maxwell's equations:


\subsubsection{Resistivity Note}

While when using the ideal MHD equations we make many simplifications, there 
is one that we should note perticularly given what we are trying to model. 
Resistivity is a phenomenon

Talk about AC reactors, acceleration of electrons important 

Explain why perturbation theory is justified for resitivity

Allegory below


A perhaps more intuitive analogy is that of swimming. Imagine you are stationary under water 
in a pool, floating, and you look at your arm extended out in front of you. 
Focus on the feeling of the water moving around your arm - this will emulate the resistivity 
that electrons would feel. If you move your arm abruptly, 
perhaps in a cutting motion, then you will feel the pressure of the water push against 
your arm, and perhaps you have to use noticably more force to overcome this pressure.
Now imagine instead that you don't abruptly cut, but instead very slowly, almost 
imperceptibly, move your arm through the water. You will notice that the feeling 
of water \emph{resisting} your arms motion will not be as noticeable. 

NOTE: perhaps it's better told from the perspective of an external observer?
Otherwise this might imply that electron speeds should differ, not that the observed force 
is different.

% 3.2.
\section{Grad-Shafranov-Helmholtz Equation}

% 3.2.1.
\subsection{Helmholtz Equations}

What is a Helmholtz equation 

% 3.2.2.
\subsection{GSH Equation}
How do we apply it to the Grad-Shafranov equation

The work here will largely follow the paper by Wang and Yu REFERENCE,
with some key notes and modifications. We are by now familiar with the 
GSE:

INSERT GSE

Wang provide a slightly different (though equivalent) formulation:
\begin{proposition}
    First, we will introduce some normalisation terms. Let $j_\phi = \frac{J_{\phi} \mu_0 a}{B_0}$ be 
    the toroidal current density normalised with respect to the magnetic field, recalling that $J_{\phi}$ 
    is the toroidal current density, $\mu_0$ is free-space magnetic permeability, 
    and $B_0$ is the on-axis magnetic field strength (here, the strength of the 
    magnetic field evaluted at the centre of the cross section, where $x = a$).

    The GSE can then equivalently be stated:
    \begin{align}
        \left ( x \pdv{x} \frac{1}{x} \pdv{x} + \pdv[2]{z} \right ) \psi &= -\frac{1}{2} x^2 \dv{\beta}{\psi} - \frac{1}{2}\dv{g^2}{\psi} = -x j_{\phi} \\
        \intertext{where}
        a_1 &= -\frac{1}{2} \dv{\beta}{\psi} \\
        a_2 - \alpha^2 \psi &= -\frac{1}{2} \dv{g^2}{\psi}
    \end{align}

    Here, we have some normalised terms: $\psi(x,z) = \frac{\Psi(x,z)}{B_0 a^2}$ is the normalised poloidal magnetic flux function. The $\beta$ 
    function is given $\beta(\psi) = \frac{2 \mu_0 p(\psi)}{B_0^2}$, and $g(\psi) = \frac{F(\psi)}{B_0 a}$. 
\end{proposition} 

\begin{remark}
    We highlight here the parameter tuple $(a_1, a_2, \alpha)$. We will see soon that 
    these parameters can be used to determine our system. For now we simply note their 
    significance so that the reader (that's you!) may pay them special attention through the rest of the following working.
\end{remark}

At the boundary of a fusion reactor we expect there to be no / minimal poloidal magnetic 
flux, i.e. that the field strength dissipates as we approach the edge of the reactor. This justifies 
the boundary condition we will impose: $\psi |_b = 0$. 


% 3.2.3.
\subsection{Analytic Solution and Derivation}

% 3.3.
\section{Current Reversal Systems}

% 3.3.1.
\subsection{Current and Pressure Density Profiles}

% 3.3.2.
\subsection{Example Configurations}
