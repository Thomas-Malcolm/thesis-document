
\chapter{Numerical Model Fitting}
\label{chapter4}

% 4.1.
\section{Non-Linear Optimisation}

% 4.1.1.
\subsection{Least Squares}

\red{Standard $L_2$ norm explanation and stuff}

% 4.1.2.
\subsection{Optimisation Algorithms}

Minimising least squares.
\red{Motivate optimisation algorithms}

% 4.1.2.1
\subsubsection{The Usual Suspects}

\red{Newton's method}

\red{Gradient descent}

\red{Bounded vs non-bounded optimisation}

% 4.1.2.2.
\subsubsection{MMA Algorithm}

\red{just summarise below more or less}
\red{https://fab.cba.mit.edu/classes/865.18/design/optimization/mma.pdf}

Method of Moving Asymptotes (MMA)

Provide explanation of LD-MMA algorithm. 

% 4.2.
\section{GSH Parameter Fitting}

\red{Talk about why we want to find $a_1, a_2$ and $\alpha$, and how we could go about 
doing that}

% 4.2.1
\subsection{Optimisation Function}

\red{Talk about data we have available}

\red{Construct the objective function}

\red{Bounds for the objective function and justification}

% 4.2.2
\subsection{Parameter Space}

\red{Parameter space a bit cooked}

\red{Show for various zooms}

\red{Talk about asymptotic behaviour with $\alpha$ (justifies MMA usage)}

Graphs showing effect of different parameters

Non-reliance on $\alpha$ (with exception of $\frac{1}{\alpha}$ where $\alpha = 0$ thing)

% 4.2.3
\subsection{Convergence Difficulties}

\red{Could be merged with above?}

Initial difficulties with normalisation. 

% 4.3
\section{Simulated Current Reversals}

\red{Have ability to solve for $a_1, a_2$ and $\alpha$, so next step is 
the ``time perturbation'' - hark back to chapter 3 here}


% 4.3.1
\subsection{Method of Reversal}


\red{Linearly vary current. Could talk about alternatives - i.e. if 
our time perturbation had not been linear but had been trigonometric (to 
fit with a larger scope AC simulation) then our current reversal method 
here could be different}

\red{Feed solved parameters into guess for next. Emphasise problem of not knowing 
the initial ones nonetheless}


% 4.3.2
\subsection{Results and Explanations}

\red{Have still frame shots of the result of that (3x3 tiles?)}

\red{Show the slow version, then the 'zoomed' in version}

\red{Highlight magnetic field topology breaking - suggests RE population generation}

\red{Comment on feasibility of the behaviour - e.g.pressure density profile follows 
magnetic field axis, but then there are things like current densities 
at the edge of the reactor}

