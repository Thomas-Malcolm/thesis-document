
\chapter{Blue Skies and Horizons}
\label{chapter6}

% 6.1
\section{Conclusions}

% 6.2.
\section{Further Work}

% 6.2.1
\subsection{Simulated Electric Field via Fake Solenoids}

In a meeting while presenting my findings to the Plasma Science group, I 
posed the question of extracting electric field information from the data 
we had available. There are many benefits to being able to describe the 
electric field for the confinement time of a plasma, however the 
most significant to our purpose is to positively identify the 
birth of runaway electrons. This information however is not readily 
available with the system we worked with.

David Pfefferle proposed a method of simulating the presence of 
an electric field instead. His idea was to introduce infinitesimal 
solenoids at the centre of magnetic islands, which would each contribute 
produce their own electric field. These would then interact, with the idea 
that the product would be an approximation to the expected 
electric field for a given state. 

The physical justification for this is that the solenoid's magnetic field is 
emulating the magnetic field produced by current densities, which 
are themselves informed by magnetic islands. A toroidal current 
will produce an electric field, including a poloidal component, which 
will influence the behaviour of runaway electrons. Thus, this approach 
effectively emulates the presence of a poloidal electric field using 
position and strength information of current densities.

This approach could utilise work done by Nicholas Bohlsen in using topological 
data analysis to identify the presence of magnetic islands from poloidal magnetic flux data in his thesis.

% 6.2.2
\subsection{Comparison with ISTTOK vloop data}

% 6.2.3
\subsection{Retrieval of }
