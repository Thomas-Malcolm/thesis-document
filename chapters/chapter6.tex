
\chapter{Blue Skies and Horizons}
\label{chapter6}

% 6.1
\section{Conclusions}

In this thesis we've built up an ability to reason about perturbations in time about 
equilibrium solutions to a variant of the Grad-Shafranov Equation. We did this with the intent 
of modelling the ISTTOK reactor's ramp down phase as a tool for identifying 
causal mechanisms for the generation of runaway electrons. Our model, using current 
density profile data provided by the ISTTOK project, is able to infer the topology 
of the poloidal magnetic field, with accompanying pressure density profile. Our 
efforts in simulating the change in poloidal magnetic field topology as the current 
density profile varies suggests there are mechanisms induced by the current inversion 
that can generate runaway electrons, providing one potential theoretical explanation 
for the observed spikes in runaway electron populations as observed by ISTTOK \cite{malaquias-matthew}.
However, we can make no concrete statements in relation to the presence of two anti-parallel 
currents in the plasma, \red{confirm this by looking at pressure density profile with magnetic field topology, 
and whether the current density is negative or not}

With that being said, there is still a lot of work that could be done to improve this 
model and compare it to literature in a more robust fashion. Some of the efforts that 
could be made to improve upon the work in this thesis are now presented.

% 6.2.
\section{Further Work}

% 6.2.1
\subsection{Simulated Electric Field via Fake Solenoids}

In a meeting while presenting my findings to the Plasma Science group, I 
posed the question of extracting electric field information from the data 
we had available. There are many benefits to being able to describe the 
electric field for the confinement time of a plasma, however the 
most significant to our purpose is to positively identify the 
birth of runaway electrons. This information however is not readily 
available with the system we worked with.

David Pfefferle proposed a method of simulating the presence of 
an electric field instead. His idea was to introduce infinitesimal 
solenoids at the centre of magnetic islands, which would each contribute 
produce their own electric field. These would then interact, with the idea 
that the product would be an approximation to the expected 
electric field for a given state. 

The physical justification for this is that the solenoid's magnetic field is 
emulating the magnetic field produced by current densities, which 
are themselves informed by magnetic islands. A toroidal current 
will produce an electric field, including a poloidal component, which 
will influence the behaviour of runaway electrons. Thus, this approach 
effectively emulates the presence of a poloidal electric field using 
position and strength information of current densities.

This approach could utilise work done by Nicholas Bohlsen in using topological 
data analysis to identify the presence of magnetic islands from poloidal magnetic flux data in his thesis.

% 6.2.2
\subsection{Comparison with ISTTOK vloop data}

% 6.2.3
\subsection{Grid Based Initial Parameter Guessing}
\red{Grid based approach for having a better initial parameter guess}

