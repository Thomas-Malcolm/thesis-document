
\chapter{Blue Skies and Horizons}
\label{chapter6}

% 6.1
\section{Conclusions}

In this thesis we've built up an ability to reason about perturbations in time about 
equilibrium solutions to a variant of the Grad-Shafranov Equation. We did this with the intent 
of modelling the ISTTOK reactor's ramp down phase as a tool for identifying 
causal mechanisms for the generation of runaway electrons. Our model, using current 
density profile data provided by the ISTTOK project, is able to infer the topology 
of the poloidal magnetic field, with accompanying pressure density profile, though the 
accuracy of the model as it stands with respect to inputted data is in question. Our 
efforts in simulating the change in poloidal magnetic field topology as the current 
density profile varies nevertheless suggests there are mechanisms induced by the current inversion 
that can generate runaway electrons, providing one potential theoretical explanation 
for the observed spikes in runaway electron populations as observed by ISTTOK \cite{malaquias-matthew}.
However, we can make no concrete statements in relation to the presence of two anti-parallel 
currents in the plasma. 

There is still a lot of work that could be done to improve this 
model and compare it to literature in a more robust fashion. Some of the efforts that 
could be made to improve upon the work in this thesis are now presented.

% 6.2.
\section{Further Work}

% 6.2.1
\subsection{Simulated Electric Field via Fake Solenoids}

In a meeting while presenting my findings to the Plasma Science group, I 
posed the question of extracting electric field information from the data 
we had available. There are many benefits to being able to describe the 
electric field for the confinement time of a plasma, however the 
most significant to our purpose is to positively identify the 
birth of runaway electrons. This information however is not readily 
available with the system we worked with.

David Pfefferle proposed a method of simulating the presence of 
an electric field instead. His idea was to introduce infinitesimal 
solenoids at the centre of magnetic islands, which would each contribute 
produce their own electric field. These would then interact, with the idea 
that the product would be an approximation to the expected 
electric field for a given state. 

The physical justification for this is that the solenoid's magnetic field is 
emulating the magnetic field produced by current densities, which 
are themselves informed by magnetic islands. A toroidal current 
will produce an electric field, including a poloidal component, which 
will influence the behaviour of runaway electrons. Thus, this approach 
effectively emulates the presence of a poloidal electric field using 
position and strength information of current densities.

This approach could utilise work done by Nicholas Bohlsen in using topological 
data analysis to identify the presence of magnetic islands from poloidal magnetic flux data in his thesis.

% 6.2.2
\subsection{Comparison with ISTTOK $v_{\text{loop}}$ data}
In the appendix of Wang's paper there are small extensions to their results if 
some slightly different assumptions are made about the model \cite{wang-analytic-solution}. If these assumptions are made, we are presented with the 
ability to both derive information about the electric field, and calculate $v_{\text{loop}}$ values. Both of these
are of particular interest to us; if we are able to retrieve information about the electric field then we can use that to reason 
about the generation of runaway electrons directly by inspecting the strength of the electric field under certain conditions. 
Additionally, with the ability to calculate $v_{\text{loop}}$ data, we would have another metric by which to test the accuracy 
of our simulation against experiment, as the ISTTOK data we have contained $v_{\text{loop}}$ information, inferred from the 
raw data provided. This would more justify us in using pressure density profile data

% 6.2.3
\subsection{Grid Based Initial Parameter Guessing}
One of the identified problems with our model, at both the simulation stage and the data fitting, was that the initial parameter 
guess is just that; a guess. Due to the parameter space we are dealing with the system is very sensitive to this initial choice, and 
so it is crucial that a suitable initial guess is chosen for the optimisation algorithm to not fall into false positives. 
As calculating residuals is relatively cheap to perform, and we know our objective function has a smooth parameter space (with 
exception for $(\alpha^2 - \lambda_{n,l}^2)^{-1} = 0$ cases), we can utilise a grid-based search approach for identifying 
what could potentially be a good initial guess. If we describe some domain $\Sigma := \Sigma_1 \times \Sigma_2 \times \Sigma_3$ such 
that $(a_1, a_2, \alpha) \in \Sigma$, then we can segment this cuboid into smaller, uniformly sized cuboids. We could then 
pick a ``representative point'' from these smaller cuboids (for simplicity, say the centre), and calculate the residual for that 
specific parameter set with respect to the provided data. Then we pick the parameter set that has the smallest residual, and 
use that parameter set as the initial guess for our optimisation algorithm.