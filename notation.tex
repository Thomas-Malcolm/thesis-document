

\chapter{Notation and terminology}\label{notation}
%\addcontentsline{toc}{chapter}{Notation and terminology}

\renewcommand{\thefootnote}{\fnsymbol{footnote}}


\

\noindent\textbf{Notation}

% adjust the lengths to suit your needs (difference of .22cm works best)

\newcommand{\nttn}[2]{\item[{\ \makebox[3.18cm][l]{#1}}]{#2}}
\begin{list}{}{ \setlength{\leftmargin}{3.4cm}
                \setlength{\labelwidth}{3.4cm}}

\nttn{$L^2(\R^n)$}{Space of $L^2$ integrable functions on $\R^n$}

\nttn{$B_0$}{On-axis magnetic field}
\nttn{$R_0$}{Major radius (Tokamak)}
\nttn{$a$}{Minor radius (Tokamak)}
\nttn{$\Psi$}{Poloidal magnetic flux function}
\nttn{$j_{\phi}$}{Toroidal current density}
\nttn{$J_{\phi}$}{Normalised toroidal current density (normalised w.r.t. $B_0$)}
\nttn{$p$}{Plasma pressure density}

\end{list}

\

\noindent\textbf{Terminology}

% adjust the lengths to suit your needs (difference of .22cm works best)

\newcommand{\term}[2]{\item[{\ \makebox[4.58cm][l]{#1}}]{#2}}
\begin{list}{}{ \setlength{\leftmargin}{4.8cm}
                \setlength{\labelwidth}{4.8cm}}


\term{GS / GSE}{Grad-Shafranov equation}
\term{GSH}{Grad-Shafranov-Helmholtz equation}
\term{MHD}{Magnetohydrodynamics}


\end{list}
